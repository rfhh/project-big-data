\documentclass[a4paper]{report}

\usepackage{amsthm}
\usepackage{anysize}
\usepackage[english]{babel}
\usepackage{comment}
\usepackage[colorlinks=true,linkcolor=blue,urlcolor=blue]{hyperref}
\usepackage{paralist}

\marginsize{2cm}{2cm}{2cm}{2cm}
\setlength{\parskip}{0.5\baselineskip}
\setlength{\parindent}{0pt}

\theoremstyle{definition}
\newtheorem{exercise}{Exercise}

\newcommand{\blankline}{\par\vspace{5mm}}
\newcommand{\tab}{\hspace*{2em}}
\newcommand{\doublequote}{\texttt{"}}
\newcommand{\singlequote}{\char13}
\newcommand{\pipe}{$\vert$}


\begin{document}

\begin{center}
	\textsc{\Large Project Big Data}
	\blankline
	
	\textbf{\large Assignment 2}
\end{center}

You will continue to work with the hue data files supplied for
Assignment~1, {\small\texttt{hue\_upload.csv}} and \texttt{\small
hue\_upload2.csv}.  The first four columns represent the \emph{row id},
\emph{user id}, \emph{event id}, and \emph{value}. Any extra columns
are irrelevant. For example, the first row of one file reads:

\texttt{\small "1";"10";"lamp\_change\_29\_mei\_2015\_19\_08\_33\_984";"OFF"}

As you can see, the event id encompasses both a description of
the event (``lamp\_change'') and the date/time (May 29, 7:08:33 pm).
The following events are considered informative:

\begin{center}
\begin{tabular}{|l|l|}
\hline
\textbf{String}		& \textbf{Description} \\
\hline
\texttt{\small lamp\_change}		& Light control via app \\
\texttt{\small nudge\_time}		& Automatic light dim time for people in experimental group \\
\texttt{\small bedtime\_tonight}	& Intended bedtime (self-reported) \\
\texttt{\small risetime}		& Rise time (self-reported) \\
\texttt{\small rise\_reason}		& Reason for rising (self-reported) \\
\texttt{\small adherence\_importance}	& Adherence (self-reported) \\
\texttt{\small fitness}			& Fitness (self-reported) \\
\hline
\end{tabular}
\end{center}

All self-reported values are entered around noon. Records with other
events may be ignored.

(60 points) The first part of this assignment is to write a Python
function \texttt{\small read\_csv\_data} that reads the data into a
Pandas DataFrame. The index should be a (date,user) tuple, where date
is stored in datetime.datetime format, see the note in the `Tips' Section.
The columns of your Pandas DataFrame should be \texttt{\small bedtime},
\texttt{\small intended\_bedtime}, \texttt{\small rise\_time},
\texttt{\small rise\_reason}, \texttt{\small fitness}, \texttt{\small
adherence\_importance} and \texttt{\small in\_experimental\_group}
(note: it is important to \emph{stick to this nomenclature}). The way
to do this is by going through the csv data line by line, and parsing
each line individually, following these requirements:

\begin{description}

\item[bedtime] should be inferred from the \texttt{\small lamp\_change}
event. For example, from the row printed above you may infer that the
person did not sleep before 7:08:33 pm. As you go through the lines
in the csv file, whenever you discover new relevant information, you
either update an existing record in the dataframe (if a record for
that day and user exists), or you create a new record (see the `Tips'
Section below). For example, if you encounter a line where user 10 turns
the light on at 9 pm and another line where he turns it off at 10 pm
(still on May 29), you update the record above to change the bedtime
to 10 pm. If someone falls asleep past midnight, the bedtime should be
stored in the record corresponding to the day before.
Again, dates and times should be stored as datetime.datetime.

\item[intended\_bedtime] should be filled in based on the \texttt{\small
bedtime\_tonight} event.  Note that \texttt{\small 1030} probably
means 10:30 in the evening. Again, dates and times should be stored
as datetime.datetime.

\item[rise\_time] The value for the column in your solution should be
obtained from the \texttt{\small risetime} event in the csv file.

\item[rise\_reason, fitness and adherence\_importance] values should
be copied from the csv file. Note that if multiple distinct values are
entered, the last should be assumed to be correct.

\item[in\_experimental\_group] should be boolean (True/False). The
default value is False, but should be changed to True if a \texttt{\small
nudge\_time} event is encountered.  If a user is in the experimental
group on one day, he is on all days.

\end{description}

(10+20 points) The second part of this assignment is to store the contents
of the DataFrame into MongoDB, and to write a function that retrieves
data from MongoDB and outputs it in a user-friendly format.

\begin{enumerate}

\item The data should be stored in the collection ``sleepdata'' in the
database ``BigData''.  Make sure to use the same column names as specified
for the DataFrame, and to define the correct primary key. See below in the
Section `Tips' for some comments about the primary key. Add the extra
columns ``date'', ``user'', ``sleep\_duration'' to facilitate sorting
the data. Here, ``sleep\_duration'' is the difference between the rise
time and the bed time.

\item The following is an example of how the
output must be presented.

\begin{tabular}{|r r r r r r r r|}
\hline
date & user & intended & actual & rise & fitn & adh & exp \\
11-06-2015 & 34 & 22:30:00 & 00:51:28 & 07:00:00 & 07:10:00 & - & 47.0 \\
11-06-2015 & 20 & 23:00:00 & 00:28:10 & 55.0 & 88.0 & x & \\
\hline
\end{tabular}

\end{enumerate}

Another 10 points are awarded for the cleanliness of your code and
the use of idiomatic Python. A total of 100 points can be obtained.
The assignment should be done in groups of two students, and must be
handed in via Blackboard on June 15 by 23:59h. Your solution should use
the provided template (\texttt{\small solution.py}). It is \emph{crucial}
that you do not alter the names and arguments of the existing functions,
although adding additional functions is recommended. The template
contains the file \texttt{\small run\_solution.py} that shows how we
run your file. Each group should hand in only one solution. If both
students submit a solution, only the first submission will be graded.
Feedback will be provided to the e-mail addresses you provide in
\texttt{\small solution.py}.


\section*{Tips}

In the assignments, there are some hurdles that are hard to jump without
further help.

\begin{itemize}
\setlength\itemsep{1mm}

\item It is important to use the datetime.datetime datatype instead of
the (seemingly more appropriate) datetime.date type, as the following
example shows. The following code \emph{does not} run as expected:

% Need pastable quotes so cannot use verbatim
\texttt{\small
idx = (datetime.date(2015,1,1),10)\\
df =
df.append(pd.Series(\{\singlequote{}bedtime\singlequote{}:None,\singlequote{}intended\_bedtime\singlequote{}:None\}, name=idx))\\
if idx in df.index:\\
\mbox{}~~~~print(\doublequote{}surprisingly, this line does not
run\doublequote{})
}

The reason is that pandas converts the
datetime.date to a pandas datetime type. Since a comparison between a date
and a datetime is always false, idx is not found in df.index. So, just stick
to the datetime.datetime data type for now (we suspect it to be a bug in
Pandas that will get fixed some day, but not soon enough for this course).
The following code (the change is in the first line) \emph{does run} as
expected:

\texttt{\small idx = (datetime.datetime(2015,1,1,0,0,0),10)\\
df =
df.append(pd.Series(\{\singlequote{}bedtime\singlequote{}:None,\singlequote{}intended\_bedtime\singlequote{}:None\}, name=idx))\\
if idx in df.index:\\
\mbox{}~~~~print(\doublequote{}This gets printed (as expected)\doublequote{})
}


\item The next hurdle is that in some configurations, calling \texttt{\small
df.set\_value} leads to
the error ``ValueError: could not convert string to float''. It turns
out to be difficult to pinpoint the exact cause of this error. As a
workaround, we advise to use the following function:

% Need pastable quotes so cannot use verbatim
\texttt{\small
def~insert\_if\_new(df,idx):\\
\mbox{}~~~~if~idx~not~in~df.index:\\
\mbox{}~~~~~~~~df~=~df.append(pd.Series(\{\singlequote{}bedtime\singlequote{}~:~float(\singlequote{}nan\singlequote{}),~\textbackslash\\
\mbox{}~~~~~~~~~~~~~~~~~~~~~~~~~~~~~~~~~~\singlequote{}intended\_bedtime\singlequote{}~:~float(\singlequote{}nan\singlequote{}),~\textbackslash\\
\mbox{}~~~~~~~~~~~~~~~~~~~~~~~~~~~~~~~~~~\singlequote{}risetime\singlequote{}~:~float(\singlequote{}nan\singlequote{}),~\textbackslash\\
\mbox{}~~~~~~~~~~~~~~~~~~~~~~~~~~~~~~~~~~\singlequote{}rise\_reason\singlequote{}~:~float(\singlequote{}nan\singlequote{}),~\textbackslash\\
\mbox{}~~~~~~~~~~~~~~~~~~~~~~~~~~~~~~~~~~\singlequote{}fitness\singlequote{}~:~float(\singlequote{}nan\singlequote{}),~\textbackslash\\
\mbox{}~~~~~~~~~~~~~~~~~~~~~~~~~~~~~~~~~~\singlequote{}adherence\_importance\singlequote{}~:~float(\singlequote{}nan\singlequote{}),~\textbackslash\\
\mbox{}~~~~~~~~~~~~~~~~~~~~~~~~~~~~~~~~~~\singlequote{}in\_experimental\_group\singlequote{}~:~False\},~\textbackslash\\
\mbox{}~~~~~~~~~~~~~~~~~~~~~~~~~~~~~~~~~name=idx))\\
\mbox{}~~~~return~df
}

Mixing float with \texttt{\small nan} ensures that Pandas does not
infer incorrect datatypes. Call this function to add a row, and use
\texttt{\small set\_value} to modify the dataframe.

\item The third hurdle is setting the primary key in mongodb. As mongodb
does not support tuples (just dicts and lists), the primary key has to be
converted from a tuple to a dict (with strings 'date' and 'user' as keys).

\item Good luck! Oh, for those in need, there is an official
\href{https://github.com/pandas-dev/pandas/blob/master/doc/cheatsheet/Pandas\_Cheat\_Sheet.pdf}{Pandas
cheat sheet}.
\end{itemize}

\end{document}
