\documentclass[a4paper]{report}

\usepackage{amsthm}
\usepackage{anysize}
\usepackage[english]{babel}
\usepackage{comment}
\usepackage[colorlinks=true,linkcolor=blue,urlcolor=blue]{hyperref}
\usepackage{paralist}

\marginsize{2cm}{2cm}{2cm}{2cm}
\setlength{\parindent}{0pt}

\theoremstyle{definition}
\newtheorem{exercise}{Exercise}

\newcommand{\blankline}{\par\vspace{5mm}}
\newcommand{\tab}{\hspace*{2em}}
\newcommand{\doublequote}{\texttt{"}}
\newcommand{\singlequote}{\char13}
\newcommand{\pipe}{$\vert$}


\begin{document}

\begin{center}
    \textsc{\Large Project Big Data}
    \blankline
    
    {\large Assignment week 4 (Report)}
\end{center}

You will continue working with data from the ``HUE bedtime procrastination
study''. A cleaned version of the hue data will be made available
(\texttt{\small hue\_week\_4\_2017.csv}), as well as another file that
contains data from the post‐study questionnaire that participants filled
out at the end of the study (\texttt{\small hue\_questionnaire.csv}). This
file contains information about

\medskip
\begin{tabular}{|l|p{0.7\textwidth}|}
\hline
gender  & 1 = male, 2 = female \\

age & \\

chronotype  & Single item (7-­‐point scale), do you consider yourself
	more of a morning [1] or an evening person? [7] \\

bp\_scale  & Dutch version of the Bedtime Procrastination Scale
\href{http://selfregulationlab.nl/wp-content/uploads/2012/04/J-Health-Psychol-2016-Kroese-853-62.pdf }{Kroese et al., 2014} \\

motivation  & Questions pertaining to personality traits related to
	procrastination. Single item (7-­‐point scale), how motivated
	were you to go to bed on time each night? (1 = not motivated;
	7 = very motivated) \\

daytime\_sleepiness  & Dutch translation of the Epworth Sleepiness Scale
	(4-­‐point scale from 0-­‐3; 8 questions, values summed)
	\href{http://www.mwjohns.com/wp-content/uploads/2009/murray\_papers/a\_new\_method\_for\_measure\_daytime\_sleepiness\_the\_epworth\_sleepiness\_scale.pdf}{ESS; Johns, 1991}. \\

self\_reported\_effectiveness  & Single item (7-­‐point scale), do you feel
	more rested since the intervention? \\
\hline
\end{tabular}
\medskip

For the final assignment, you will use Python to examine this
post‐questionnaire data in relation to the HUE data file, visualize
trends and relationships, look for correlations between factors, test
for significant differences between groups and build a regression model
to predict bedtime delay. You are required to hand in your Python code
to show that all transformations, visualizations and analyses have been
done in Python. Your Python code will not be graded, however. You will
bundle your findings in a neatly formatted final report of at most 6 pages
(cover page excluded). This report should not contain any python code. See
page 3 of this document for the structure and content that the report
should have. For the grading criteria please refer to the syllabus. The
report should be written in English.  In order to perform the analyses,
a number of transformations on the data still need to be done. To help
you along, a Python template will be made available with a recommended
structure for your Python code. The following steps must be implemented.

\begin{itemize}
\item Read the hue data file and the questionnaire data file into two
separate pandas DataFrames.
\item Create a new DataFrame that contains the following Series:

\begin{tabular}{l p{0.6\textwidth}}
ID & Participant ID, \\
group & Participant group (1 for experimental, 0 for control), \\
delay\_nights & The number of nights a participant delayed their bedtime
	(range: 0-12),  \\
delay\_time  &
	The mean time in seconds a participant delayed their bedtime
	(total delay in seconds, divided by the number of observations
	measured for each individual, rounded to nearest second). \\
\end{tabular}

Set the index of this new DataFrame to 'ID'. Note that there should only
be a single row per participant ID.

\item Fill this new DataFrame by transforming data from the DataFrame
about participants' bedtimes (from the hue data file).

\item Merge this new DataFrame with the post-­‐questionnaire data and
store the resulting DataFrame in a new variable. Perform this joining
operation of the two DataFrames in such a way that the resulting DataFrame
only contains IDs that were present in both datasets.

\item Use the scipy.stats package to calculate correlations between the
	following sets of determinants:
	\begin{itemize}
	\item bedtime procrastination scale (bp\_scale; a personality
		trait) and mean time spent delaying bedtime. Use the
		``Pearson correlation tests'' to calculate the
		correlation.
	\item age and mean time spent delaying bedtime. Use the ``Kendall
		rank correlation test'' to calculate the correlation.
	\item mean time spent delaying bedtime and daytime sleepiness. Use
		the ``Pearson correlation test'' to calculate the
		correlation.
	\end{itemize}

\item Use the
\href{http://docs.scipy.org/doc/scipy/reference/stats.html}{scipy.stats}
package to determine whether there are significant differences between
the experimental group and the control group in terms of:
	\begin{itemize}
	\item the number of nights participants delayed their bedtime
	\item the time participants spent in bed each night
	\item the mean time participants spent delaying their bedtime
	\end{itemize}
Use knowledge gained in the course `Statistics' to determine
which statistical test is appropriate: the t-test or the Wilcoxon
rank-sum test. Explain your choice in the Data analyses section of
your report.

\item For bonus points: When interpreting the findings from the
comparisons, adjust the significance level (alpha) to account for an
inflation of Type I errors due to performing multiple comparisons. For
example, you could use a Bonferroni correction (feel free to consult a
reference work on how to adjust the significance level to counteract
the inflation of Type I errors).

\item Formulate and concisely argue for a hypothesis about which factor
or factors (max. three) you believe would best predict delay\_time. Write
your hypothesis down in the Data analyses section of your report. Note
that you should theorize about why you think these factors might be good
predictors, and not use stepwise regression to identify the strongest
predictors.

\item Use \texttt{\small statsmodels.api} to build a regression model
that uses your three hypothesized determinants to predict delay\_time
(see page 1 of this document). Make sure that delay\_time is not included
in the list of predictors. Again, do not use stepwise regression, but
use the regression model to test your hypothesis.

\item Create three distinct, meaningful, well-crafted visualizations
that either provide insight into the data, or help support your
conclusions. This means creating three different kinds of plots (not
three boxplots, or three scatterplots for example).

\item Interpret and discuss your findings in the Discussion section of
your report.

\item Write a succinct conclusion.
\end{itemize}

Report your findings in a separate document. This document should contain
the following sections:

\begin{description}
\item[Introduction (100 words)]
	Describe concisely what your report is about. State the research
	question
	or research questions with which the report is concerned.

\item[Data description and exploration (500 words)]
	Describe the variables in your final, merged DataFrame: what does each
	value represent?

	For each variable, report appropriate descriptives (e.g., mean and
	standard deviation, or frequencies, etc.) Mention strange values, and
	possible outliers.

	This is a good place to include one or more visualizations that help the
	reader to better understand the data.

\item[Data analyses (400 words)]

	Describe the statistical analyses that you performed, and explain why
	those were the appropriate tests for the data. Report your findings,
	but don't discuss them yet. This is another place where visualizations can provide insight into the findings.

\item[Discussion (400 words)]

	Discuss your findings. Are correlations high or low? Does that make sense? Are differences significant? How robust do you think your findings are?

\item[Conclusion (100 words)]

	What conclusions could be drawn from your analyses? Prefer nuanced statements to bold, sweeping claims.

\end{description}

The final assignment should be done in the same groups as in the
previous weeks, and must be handed in via Canvas on June 29 by
23:59 hours. There will be two separate assignments on Canvas, one
for your Python code and one for your report. You must submit one file
to each assignment (one python file and one PDF document, respectively).

The report should be formatted as a PDF document and must have the
structure as described above. The word count per section is the maximum
number of words allowed, excluding tables, captions and figures (note
that this word count is strict). A table of contents is not necessary
and should be left out. The report contains page numbers on each page
in the bottom right corner. The report must have a front page containing
the names and student numbers of the authors, the title of the report,
the name of the course, and the date of submission.

Each group should hand in only one solution. If two Python files or two
PDF files are submitted, only the first submission will be graded.

\end{document}
