\documentclass[a4paper]{report}

\usepackage{amsthm}
\usepackage{anysize}
\usepackage[english]{babel}
\usepackage{comment}
\usepackage[colorlinks=true,linkcolor=blue,urlcolor=blue]{hyperref}
\usepackage{paralist}

\marginsize{2cm}{2cm}{2cm}{2cm}

\theoremstyle{definition}
\newtheorem{exercise}{Exercise}

\newcommand{\blankline}{\par\vspace{5mm}}
\newcommand{\tab}{\hspace*{2em}}
\newcommand{\doublequote}{\texttt{"}}
\newcommand{\singlequote}{\char13}
\newcommand{\pipe}{$\vert$}


\begin{document}

\begin{center}
    \textsc{\Large Project Big Data}
    \blankline
    
    {\large Assignment 3}
\end{center}

For each of the following assignments, use the hue dataset supplied on
Canvas (\texttt{\small hue\_week\_3\_2017.csv}). Furthermore, assume
that there are multiple threads handling the mapping phase, but that
there is only one thread in charge of reducing. This means that if you
need to use global variables, you should do so only in the reducer. In
addition, assume that the end of the input will be signaled by an empty
row (delimiters only, no values). Finally, make sure that each function
writes its output to STDOUT via the \texttt{\small print()} function.

\begin{exercise}
(10 points) Write a MapReduce algorithm that counts and outputs the
total number of times the fitness value is strictly higher than 50. The
expected output is a single integer.
\end{exercise}

\begin{exercise}
(10 points) Write a MapReduce algorithm that calculates the mean fitness
per participant. Do not use any statistical packages to calculate
the mean. The expected output is one line per participant, containing
the participant’s ID and the mean of his or her fitness, separated by
a tab (“\t”). The outputted lines do not have to be sorted.
\end{exercise}

\begin{exercise}
(20 points) Write a MapReduce algorithm that calculates the mean bedtime
per participant. So, if someone goes to bed at 23.00h one night,
and at 00.30h the next, the mean bedtime for those two nights would
be 23:45h. The expected output is one line per participant, containing
the participant’s ID and the mean of his or her bedtimes, separated
by a tab (“\t”). The outputted lines do not have to be sorted.
\end{exercise}

\begin{exercise}
(15 points) Write a MapReduce algorithm that outputs the top
five ‘biggest bedtime procrastinators’. Participants cannot
“compensate” by going to bed earlier on some other nights. Finally,
assume that the reducer cannot accumulate all the data before sorting
due to memory restrictions; so implement a custom, memory-­‐efficient
sorting method in the reducer that keeps the top five up-­‐to-­‐date
for each incoming line of data. The expected output is five lines long,
with each line containing a participant’s ID and the number of seconds
he or she procrastinated.
\end{exercise}

The following three assignments pertain to parallelization through the
use of a (FIFO) queue.

\begin{exercise}
(15 points) Implement the function \texttt{\small instantiate\_queue()} that
instantiates a queue with a maximum size of 30 (the Queue class is
already imported in the template). In addition, implement the function
\texttt{\small consume\_data\_stream(queue)} using the “requests” library to connect to
the data stream \texttt{\small http://stream.meetup.com/2/rsvps} (make sure to add the
argument stream=True to the \texttt{\small requests.get()} call). If “requests” is
not available, use “pip/pip3/pip3.6 install requests” to install it.
The function should parse 30 JSON entries, add each entry to the queue,
and print the current size of the queue. After 30 entries, it should
stop consuming data.
\end{exercise}

\begin{exercise}
(10 points) Implement the \texttt{\small function process\_queue(queue)} that repeatedly
gets an element from the queue. If the element is a dictionary that
contains the key ‘venue’, it should print the coordinates of the
venue (for example: (47.423569, 8.56151)). If the element from the
queue is None, the function should return (otherwise the function might
hang forever).
\end{exercise}

\begin{exercise}
(10 points) Implement the function \texttt{\small main()} that

\begin{itemize}
\item calls \texttt{\small instantiate\_queue()} once and stores the returned queue object
in a variable queue,
\item calls \texttt{\small consume\_data\_stream(queue)} once and blocks (waits) until the
queue is filled,
\item instantiates three “worker” Threads that have as its target
the function process\_queue with argument queue,
\item blocks (waits) until the queue is empty,
\item puts None into the queue three times,
\item joins the worker threads (and exits).
\end{itemize}
\end{exercise}

Another 10 points are awarded for the cleanliness of your code and the
use of idiomatic Python. A total of 100 points can be obtained.

The assignment should be done in groups of two students, and must be
handed in via Blackboard on June 26 by 14:00 hours. The file must
have the same structure as the template (including the disclosure
and the function names). Adding additional functions is recommended.
Each group should hand in only one solution.

\end{document}
