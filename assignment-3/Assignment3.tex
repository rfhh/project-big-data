\documentclass[a4paper]{report}

\usepackage{amsthm}
\usepackage{anysize}
\usepackage[english]{babel}
\usepackage{comment}
\usepackage[colorlinks=true,linkcolor=blue,urlcolor=blue]{hyperref}
\usepackage{paralist}

\marginsize{2cm}{2cm}{2cm}{2cm}

\theoremstyle{definition}
\newtheorem{exercise}{Exercise}

\newcommand{\blankline}{\par\vspace{5mm}}
\newcommand{\tab}{\hspace*{2em}}
\newcommand{\doublequote}{\texttt{"}}
\newcommand{\singlequote}{\char13}
\newcommand{\pipe}{$\vert$}


\begin{document}
	
	\begin{center}
		\textsc{\Large Project Big Data}
		\blankline
		
		\textbf{\large Assignment 3}
	\end{center}
	
	\noindent For each of the following assignments, use the hue dataset supplied on Canvas (\texttt{\small hue\_week\_3\_2017.csv}). Furthermore, assume	that there are multiple threads handling the mapping phase, but that there is only one thread in charge of reducing. This means that if you need to use global variables, you should do so only in the reducer. In addition, assume that the end of the input will be signaled by an empty row (delimiters only, no values). Finally, make sure that each function writes its output to the standard output via the \texttt{\small print()} function.
	
	\begin{exercise}
		(10 points) Write a MapReduce algorithm that counts and outputs the total number of times the fitness value is strictly higher than 50. The expected output is a single integer.
	\end{exercise}
	
	\begin{exercise}
		(10 points) Write a MapReduce algorithm that calculates the mean fitness per participant. Do not use any statistical packages to calculate the mean. The expected output is one line per participant, containing the participant’s ID and the mean of his or her fitness, separated by a tab (\texttt{\doublequote{}$\backslash{}$t\doublequote{}}). The outputted lines do not have to be sorted.
	\end{exercise}
	
	\begin{exercise}
		(20 points) Write a MapReduce algorithm that calculates the mean bedtime per participant. So, if someone goes to bed at 23.00h one night, and at 00.30h the next, the mean bedtime for those two nights would be 23:45h. The expected output is one line per participant, containing the participant’s ID and the mean of his or her bedtimes, separated by a tab (\texttt{\doublequote{}$\backslash{}$t\doublequote{}}). The outputted lines do not have to be sorted.
	\end{exercise}
	
	\begin{exercise}
		(15 points) Write a MapReduce algorithm that outputs the top five ‘biggest bedtime procrastinators’. Participants cannot “compensate” by going to bed earlier on some other nights. Finally, assume that the reducer cannot accumulate all the data before sorting due to memory restrictions; so implement a custom, memory-efficient sorting method in the reducer that keeps the top five up to date for each incoming line of data. The expected output is five lines long, with each line containing a participant’s ID and the number of seconds he or she procrastinated.
	\end{exercise}
	
	\noindent The following three assignments pertain to parallelization through the use of a (FIFO) queue.
	
	\begin{exercise}
		(15 points) Implement the function \texttt{instantiate\_queue()} that initializes a queue with a maximum size of 30. In addition, implement the function \texttt{consume\_data\_stream(queue)} using the “requests” library to connect to the data stream \texttt{https://stream.meetup.com/2/rsvps} (add the argument \texttt{stream = True} to the \texttt{requests.get()} call). If “requests” is not available, use “\texttt{pip/pip3/pip3.6 install requests}” to install it.
		
		The function should parse 30 JSON entries, add each entry to the queue,	and print the current size of the queue. After 30 entries, it should stop consuming data.
	\end{exercise}
	
	\begin{exercise}
		(10 points) Implement the \texttt{\small function process\_queue(queue)} that repeatedly gets an element from the queue. If the element is a dictionary that contains the key ‘venue’, it should print the coordinates of the venue (for example: \texttt{(47.423569, 8.56151)}). If the element from the queue is \texttt{None}, the function should return (otherwise the function might hang forever).
	\end{exercise}
	
	\begin{exercise}
		(10 points) Implement the function \texttt{main()} that
		
		\begin{itemize}
			\item calls \texttt{instantiate\_queue()} once and stores the returned queue object in a variable queue,
			\item calls \texttt{consume\_data\_stream(queue)} once and blocks (waits) until the queue is filled,
			\item instantiates three “worker” Threads that have as its target the function process\_queue with argument queue,
			\item blocks (waits) until the queue is empty,
			\item puts None into the queue three times,
			\item joins the worker threads (and exits).
		\end{itemize}
	\end{exercise}
	
	\noindent Another 10 points are awarded for the cleanliness of your code and the use of idiomatic Python. A total of 100 points can be obtained.
	
	The assignment should be done in groups of two students, and must be handed in via Canvas on June 24 by 23:59. The file must have the same structure as the template (including the disclosure and the function names). Adding additional functions is recommended. Each group should hand in only one solution.


\section*{Frequently asked questions}

\paragraph{How do I run the program from the command line?}
Open the command line and go to the folder where your python file \texttt{template\_assignment3.py} is located. Make sure \texttt{hue\_week\_3\_2017.csv} is in this same folder. Then, use the command:\\

\noindent \texttt{python <scriptname> <datafile> <mapper function> <reducer function>}\\

\noindent so for example:\\

\noindent \texttt{python template\_assignment3.py hue\_week\_3\_2017.csv mapper1 reducer1}\\

\noindent On Mac, it might be that \texttt{python} needs to be replaced with \texttt{python3}. With this command, you indicate you will use the mapper and reduced functions, \texttt{mapper1()} and \texttt{reducer1()}, respectively.

\paragraph{When using the command, I get the error: 'python' is not recognized as an internal or external command, operable program or batch file.}
If you get this error (probably on Windows), then the terminal / command line does not know where Python is located. To achieve this, the folder where Python is installed (e.g., \texttt{C:$\backslash{}$Users$\backslash{}$Irving$\backslash{}$WinPython$\backslash{}$python-3.6.5.amd64}) needs to be added to the Path. Make sure this folder contains \texttt{python.exe}.

Recall that, to add a folder to the Path, press Windows Key + Pause, go to advanced system settings, and then to environment variables (Dutch: omgevingsvariabelen) in the advanced tab.

\paragraph{When running the program, it gives the error that the package `queue' cannot be found.}
If you get this error, you probably have installed Python 2 already on your computer. If so, whenever you use the command:\\

\noindent \texttt{python <scriptname> <datafile> <mapper function> <reducer function>}\\

\noindent the command line uses Python 2 instead of Python 3, and Python 2 does not have the queue package. To make sure the command line uses Python 3 instead of 2, go to the folder where Python is installed  (e.g., \texttt{C:$\backslash{}$Users$\backslash{}$Irving$\backslash{}$WinPython$\backslash{}$python-3.6.5.amd64}) and rename \texttt{python.exe} to \texttt{python3.exe}. Use \texttt{python3} instead of \texttt{python} in the command line.

\paragraph{What happens in the example map and reduce functions, \texttt{mapper0()} and \texttt{reducer0()}, that are provided in the template?}
In this example, we count the number of occurrences of the Strings `True' and `False' in a file using a MapReduce algorithm. During the mapping phase, lines are processed one by one, and outputs key-value pairs of which the keys equal one of the two Strings. 

The values in this example always equal one, but from exercise 2 onwards, you will notice that the values are less straightforward (they can consist of multiple values per key). At the end of the mapping phase, the output is of the form:\\

\noindent \texttt{False~~~1}\\
\noindent \texttt{False~~~1}\\
\noindent \texttt{True~~~~1}\\
\noindent \texttt{False~~~1}\\
\noindent \texttt{...}\\
\noindent \texttt{True~~~~1}\\

\noindent This output (everything printed in the mapper phase) is not printed to the standard output in the terminal, but passed to the next phase, which we have arranged for you in the provided template (the code to do this, starts under the line ``
\texttt{\# Include these lines without modifications}''). However, this output needs to be sorted, but the (straightforward) sorting phase has been done for you in the line:\\

\noindent \texttt{for index, line in enumerate(sorted(mapper\_lines)):} \\

\noindent This sorted output, is the input for the reducer. The reducer iterates efficiently through the sorted output and sums over all values equal to `True' and `False'. Note that this could not be done in this efficient manner if the input was not sorted. To conclude, when running the command:\\

\noindent \texttt{python template\_assignment3.py hue\_week\_3\_2017.csv mapper0 reducer0}\\

\noindent it should give the following output:\\

\noindent \texttt{False~~~377}\\
\noindent \texttt{True~~~~452}\\

\paragraph{For Exercise 3, how do I take the average of bedtimes? If someone goes to bed at, say 12:45 in the afternoon, does this mean the user went to bed very late or early?}
We realize that this exercise (designed from the teachers of last year) is ambiguous and feels like it needs more specification (in particular, what is the time boundary when you may assume someone the user goes to bed either very early or very late). For this reason, you may just take the average of bedtimes. Indeed, the average of 23:59 and 00:01 becomes 12:00 in the afternoon which does not make sense, but due to the fact that some information in this exercise is missing, we accept such a solution. Don't mind the counterintuitive interpretation of the result; the difficulty in this exercise is the implementation of the MapReduce algorithm, combined with the converting of Strings to times and calculating with them.

If you did implement a solution already where you did take some boundary by yourselves (e.g., if the user goes to bed between 00:00 and 06:00, add an extra day on to the bedtime), then you don't have to change this and we won't deduct any points for any interpretation.

\paragraph{For Exercise 4, how is the biggest bedtime procrastinator defined? Do you check for the maximum, average, or sum of procrastinations per user?}
Also in this exercise, we accept multiple all three answers for the same reason (for those who are curious, it was actually the total procrastination that was intended). All three possible interpretations are about equally hard to implement anyway, as the difficulty lies in the implementation of the MapReduce algorithm, combined with an efficient way to keep track of the top five procrastinators anyway.

\paragraph{For Exercise 5, my program goes stuck whenever I use the request call from the url provided and I'm not getting any data. How do I solve this?}
We noticed that on some computers, requesting the stream from the given url indeed does not work, while the exact same code does work on other computers. We believe this is due to some safety settings of computers, of virus scanners, that block specific url's. In two cases that we have found, it could be solved by adjusting the url from:\\

\noindent \texttt{http://stream.meetup.com/2/rsvps}\\

\noindent to\\

\noindent \texttt{https://stream.meetup.com/2/rsvps}. If the problem still persists, let us know.


\end{document}